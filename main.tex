\documentclass[a4paper,12pt]{article}

%\usepackage{geometry}
%\geometry{
%    a4paper,
%    total={170mm,257mm},
%    left=20mm,
%    top=20mm
%}

\usepackage{polski}
\usepackage[utf8]{inputenc}

\usepackage{amsfonts}

\usepackage{physics}

\usepackage{hyperref}

\usepackage{titlesec}
\newcommand{\sectionbreak}{\clearpage}
\newcommand{\subsectionbreak}{\clearpage}
\newcommand{\subsubsectionbreak}{\clearpage}

\newtheorem{definition}{Definicja}
\newtheorem{theorem}{Twierdzenie}
\newtheorem{example}{Przykład}

\title{Optymalizacja półokreślona w wykrywaniu splątania kwantowego}
\date{}

\begin{document}

\maketitle
\tableofcontents

\section{Wprowadzenie do informatyki kwantowej}

\subsection{Formalizm algebraiczny}

\subsubsection{Przestrzeń Hilberta}

Weźmy przestrzeń liniową wymiaru $n$ nad $\mathbb{C}$ i wyposażmy ją w iloczyn skalarny. Tak powstałą strukturę nazwiemy \textit{przestrzenią Hilberta} (ogólnie, tzn. nie skonkretyzowaną w żaden sposób przestrzeń Hilberta oznaczać będziemy $\mathcal{H}$). Element (wektor) tej przestrzeni oznaczamy $\ket{\psi}$, natomiast iloczyn skalarny pomiędzy dwoma wektorami $\ket{\psi}, \ket{\phi} \in \mathcal{H}$ oznaczymy $\bra{\psi}\ket{\phi}$ (taka konwencja określana jest mianem \textit{notacji Diraca}). Przypomnijmy jeszcze, że na przestrzeni Hilberta indukowana jest za pomocą iloczynu skalarnego \textit{norma} wektora, standardowo $||\psi|| \equiv \sqrt{\bra{\psi}\ket{\psi}}$.

\subsubsection{Baza kanoniczna}

Jako $\mathcal{B} \equiv \{\ket{i}_{i\in I}\}$ oznaczmy bazę przestrzeni $\mathcal{H}$. Zakładamy też, że $\mathcal{B}$ stanowi ortonormalny układ wektorów -- w przestrzeniach o skończonym wymiarze jesteśmy bowiem w stanie zastosować procedurę Grama-Schmidta. Dla ustalenia uwagi przyjmijmy $I \equiv \{0,1, \ldots , n - 1 \}$ -- przy takich oznaczeniach bazę $\mathcal{B}$ nazywamy bazą \textit{kanoniczną}. Przyjmuje się, że wektory bazy kanonicznej oznaczamy

$$
    \ket{i} \equiv
    \begin{pmatrix}
        0 \\
        \vdots \\
        0 \\
        1 \\
        0 \\
        \vdots \\
        0
    \end{pmatrix},
$$

gdzie nad współczynnikiem $1$ znajduje się $i$ zer.

\subsubsection{Współczynniki wektora w bazie ortonormalnej}

Przypomnimy obecnie wzór na współczynniki wektora w rozwinięciu w bazie $\mathcal{B}$, która zgodnie z założeniem stanowi układ ortonormalny, tzn. 

$$
    \bra{i}\ket{j} = \delta_{ij},
$$

gdzie dwuargumentowa funkcja $\delta_{ij}$, zwana \textit{deltą Kroneckera}:

$$
    \delta_{ij} \equiv
    \begin{cases}
        1, \text{dla } i = j \\
        0, \text{dla } i \neq j
    \end{cases}
$$

Niech więc dowolnie ustalony wektor bazowy $\ket{\alpha}$ zostanie przemnożony skalarnie przez wektor $\ket{\psi}$. Pamiętając o własnościach iloczynu skalarnego (liniowość w drugim składniku) napiszemy 

$$
    \bra{\alpha}\ket{\psi} = \bra{\alpha}\left(\sum\limits_{i}\psi_{i}\ket{i}\right) = \sum\limits_{i}\psi_{i} \bra{\alpha}\ket{i} = \sum\limits_{i}\psi_{i}\delta_{\alpha i}=\psi_{\alpha}
$$

tzn.

$$
    \psi_{\alpha}=\bra{\alpha}\ket{\psi}.
$$

\subsubsection{Iloczyn skalarny -- równoważne podejście}

Iloczyn skalarny wektorów $\ket{\psi}, \ket{\phi} \in \mathcal{H}$ można wyznaczyć wprost z definicji funkcji $\bra{\psi}\ket{\phi}$, a także za pomocą sumy współczynników tych wektorów. Dla

$$
    \ket{\psi} = \sum\limits_{i\in I}\psi_{i}\ket{i}
$$

oraz

$$
    \ket{\phi} = \sum\limits_{i\in I}\phi_{i}\ket{i}
$$

mamy bowiem

$$
    \bra{\psi}\ket{\phi} = \left(\sum\limits_{i \in I}\psi_{i}\bra{i}\right)\left(\sum\limits_{j\in I}\phi_{j}\ket{j}\right) = \sum\limits_{i, j \in I}\psi_{i} ^ {*}\phi_{j}\bra{i}\ket{j}=\sum\limits_{i \in I}\psi_{i}^{*}\phi_{i}
$$

\subsubsection{Iloczyn tensorowy}

\begin{definition}[Iloczyn tensorowy wektorów]
    Dla danych przestrzeni Hilberta $\mathcal{H}_A$ oraz $\mathcal{H}_B$ działanie mnożenia tensorowego wektorów z $\mathcal{H}_A$ z wektorami z $\mathcal{H}_B$ definiuje się jako funkcję $\otimes: \mathcal{H}_A \times \mathcal{H}_B \rightarrow \mathcal{H}_{AB}$, mającą własność liniowości w obu swoich składnikach, tzn.

    \begin{enumerate}
        \item \textbf{Jednorodność w obu składnikach.}
            Dla każdego skalara $z \in \mathbb{C}$ i dla każdych wektorów $A \in \mathcal{H}_A$ oraz $B \in \mathcal{H}_B$

            $$
                z (\ket{A} \otimes \ket{B}) = (z \ket{A}) \otimes \ket{B} = \ket{A} \otimes (z \ket{B})
            $$
        \item \textbf{Addytywność w pierwszym składniku.}
            Dla każdych wektorów $\ket{A_1}, \ket{A_2} \in \mathcal {H}_A$ oraz $\ket{B} \in \mathcal{H}_B$

            $$
                (\ket{A_1} + \ket{A_2}) \otimes \ket{B} = \ket{A_1} \otimes \ket{B} + \ket{A_2} \otimes \ket{B}
            $$
        \item \textbf{Addytywność w drugim składniku.}
            Dla każdych wektorów $\ket{A} \in \mathcal{H}_A$ oraz $\ket{B_1}, \ket{B_2} \in \mathcal{H}_B$

            $$
                \ket{A} \otimes (\ket{B_1} + \ket{B_2}) = \ket{A} \otimes \ket{B_1} + \ket{A} \otimes \ket{B_2}
            $$
    \end{enumerate}

\end{definition}

\begin{definition}[Iloczyn tensorowy przestrzeni]
    Iloczynem tensorowym przestrzeni Hilberta $\mathcal{H}_A$ (z ortonormalną bazą $\ket{i^{A}}$) i $\mathcal{H}_B$ (z ortonormalną bazą $\ket{j^{B}}$) nazywamy przestrzeń $\mathcal{H}_{AB} \equiv \mathcal{H}_A \otimes \mathcal{H}_B$, której elementy stanowią wszystkie wektory postaci

    \begin{equation}
        \label{equation:product}
        \ket{C} \equiv \ket{A} \otimes \ket{B}, \textrm{dla } \ket{A} \in \mathcal{H}_A, \ket{B} \in \mathcal{H}_B.
    \end{equation}

    Dodatkowo, wektory tej przestrzeni stanowią z definicji wszystkie kombinacje liniowe układu wektorów $\left|i^{A}\right\rangle\otimes\left|j^{B}\right\rangle$, tzn. zbiór wektorów postaci

    \begin{equation}
        \label{equation:entangled}
        \ket{C} = \sum\limits_{ij} c_{ij} \ket{i ^ {A}} \otimes \ket{j ^ {B}}.
    \end{equation}
\end{definition}

\begin{definition}[Stan splątany]
    Stan $\ket{\psi} \in \mathcal{H}_{AB}$ nazywamy splątanym wtedy, gdy nie da się go przedstawić w postaci (\ref{equation:product}). Jest on wówczas stanem o postaci (\ref{equation:entangled}).
\end{definition}

\subsubsection{Iloczyn Kroneckera}

\begin{definition}[Iloczyn Kroneckera]
    Dla macierzy $A \in M_{p \times q}(\mathbb{C})$ i $B \in M_{r\times s}(\mathbb{C})$ ich iloczyn Kroneckera zdefiniowany jest jako macierz $C \in M_{pr\times qs}(\mathbb{C})$ dana wzorem

    $$
        C \equiv A \otimes B =
        \begin{pmatrix}
            A_{00} B & A_{01} B & \ldots & A_{0, q - 1} B \\
            A_{10} B & A_{11} B & \ldots & A_{1, q - 1} B \\
            \vdots & \vdots & \vdots & \vdots \\
            A_{p - 1, 0} B & A_{p - 1, 1} B & \ldots & A_{p - 1, q - 1} B
        \end{pmatrix}.
    $$
\end{definition}

\begin{example}[Iloczyn Kroneckera]
    Dla danych macierzy

    $$
        A =
        \begin{pmatrix}
            a & b \\
            c & d
        \end{pmatrix}
    $$

    $$
        B =
        \begin{pmatrix}
            e & f & g \\
            h & i & j
        \end{pmatrix}
    $$

    ich iloczyn Kroneckera przyjmuje wartość

    $$
        C =
        \begin{pmatrix}
            a B & b B \\
            c B & d B
        \end{pmatrix}
        =
        \begin{pmatrix}
            a e & a f & a g & b e & b f & b g \\
            a h & a i & a j & b h & b i & b j \\
            c e & c f & c g & d e & d f & d g \\
            c h & c i & c j & d h & d i & d j
        \end{pmatrix}
    $$
\end{example}

\subsubsection{Przestrzeń Hilberta-Schmidta}

\begin{definition}[Przestrzeń Hilberta-Schmidta]
    Dla danej przestrzeni Hilberta $\mathcal{H}$ z bazą $\mathcal{B}$ wprowadzamy przestrzeń operatorów liniowych $A: \mathcal{H} \rightarrow \mathcal{H}$ z działaniem standardowego dodawania dwóch operatorów i działaniem mnożenia operatora przez liczbę zespoloną. Przestrzeń taka jest przestrzenią liniową. Wyposażamy ją w działanie dwuargumentowe postaci

    $$
        \bra{A}\ket{B} \equiv \textbf{Tr}[A ^ {\dag} B].
    $$

    Łatwo pokazać, że tak zdefiniowana funkcja jest iloczynem skalarnym (nazywamy ją iloczynem skalarnym Hilberta--Schmidta) -- przestrzeń operatorów rozszerzyliśmy wobec tego do przestrzeni Hilberta. Nazywamy ją przestrzenią Hilberta-Schmidta sprzężoną z przestrzenią $\mathcal{H}$ i oznaczamy $\mathcal{HS}$.
\end{definition}

\subsubsection{Operatory hermitowskie}

\begin{definition}[Operator hermitowski]
    Operator liniowy $A$ nazywamy hermitowskim (samosprzężonym), gdy

    $$
        A = A ^ {\dag}
    $$
\end{definition}

\begin{theorem}
    Wartości własne operatora hermitowskiego są rzeczywiste.
\end{theorem}

\subsection{Rozkład spektralny operatora}

\begin{definition}[Operator normalny]
    Operator liniowy $A$ nazywamy normalnym gdy

    $$
        A ^ {\dag} A = A A ^ {\dag}
    $$
\end{definition}

\begin{theorem}[Twierdzenie Spektralne]
    Operator liniowy ma rozkład spektralny $\Leftrightarrow$ jest normalny.
\end{theorem}

\subsection{Definicje}

\begin{definition}[Komutator]
    Dla operatorów liniowych $A$, $B$ ich komutator definiuje się jako operator

    $$
        [A, B] \equiv A B - B A.
    $$

    Ponadto, jeżeli $[A, B]$ jest operatorem zerowym to mówimy, że operatory $A$ i $B$ komutują.
\end{definition}

\begin{definition}[Elementy macierzowe operatora]
    Elementem macierzowym w bazie $\mathcal{B}$ operatora liniowego $A$ nazywamy liczbę

    $$
        A_{ij} \equiv \bra{i} A \ket{j},
    $$

    gdzie $\ket{i}, \ket{j} \in \mathcal{B}$. Ponadto, elementy macierzowe postaci $A_{ii}$ nazywamy elementami diagonalnymi operatora $A$.
\end{definition}

Operator $A$ można zapisać w następujący sposób:

    $$
        A = \sum\limits_{i, j \in I} A_{ij} \ket{i} \bra{j}.
    $$

\subsubsection{Ślad operatora}

\begin{definition}[Ślad operatora]
    Śladem operatora liniowego $A$ nazywamy sumę jego elementów diagonalnych:

    $$
        \textbf{Tr}[A] \equiv \sum\limits_{i \in I} A_{ii}.
    $$
\end{definition}

\begin{theorem}[Ślad operatora jest cykliczny]
    Zachodzi

    $$
        \textbf{Tr}[AB] = \textbf{Tr}[BA]
    $$
\end{theorem}

\subsubsection{Macierze Pauliego}

\begin{definition}[Macierze Pauliego]
    $$
        \sigma_{x} =
        \begin{pmatrix}
            0 & 1 \\
            1 & 0
        \end{pmatrix}
    $$

    $$
        \sigma_{y} =
        \begin{pmatrix}
            0 & -i \\
            i & 0
        \end{pmatrix}
    $$

    $$
        \sigma_{z} =
        \begin{pmatrix}
            1 & 0 \\
            0 & -1
        \end{pmatrix}
    $$
\end{definition}

\subsubsection{Operator von Neumanna}

\begin{definition}[Operator binarny]
    Operator liniowy $A$ nazywamy binarnym, gdy $A$ jest hermitowski i $A ^ 2 = I$.
\end{definition}

\begin{definition}[Operator von Neumanna]
    Jeżeli $\hat{v} \in \mathbb{R} ^ 3$ jest wektorem jednostkowym, to operatorem von Neumanna (operatorem pomiaru spinu wzdłuż osi $\hat{v}$) nazywamy operator działający na przestrzeni $V ^ {2} (\mathbb{C})$ dany wzorem

    $$
        \hat{v} \cdot \vec{\sigma} \equiv \sum\limits_{i = 1} ^ {3} v_{i} \sigma_{i} =
        \begin{pmatrix}
            v_3 & v_{1} - i v_{2} \\
            v_{1} + i v_{2} & -v_3
        \end{pmatrix}.
    $$
\end{definition}

\begin{theorem}
    Jeżeli $\hat{a} \cdot \vec{\sigma}, \hat{b} \cdot \vec{\sigma}$ są operatorami von Neumanna, to

    $$
        (\hat{a} \cdot \vec{\sigma}) (\hat{b} \cdot \vec{\sigma}) = \left(\hat{a} \cdot \hat{b} \right) I + i \left(\hat{a} \times \hat{b}\right) \cdot \vec{\sigma}.
    $$
\end{theorem}

\begin{theorem}
    Operator von Neumanna $\hat{v} \cdot \vec{\sigma}$ jest operatorem hermitowskim. Ponadto łatwo sprawdzić, że $\left(\hat{v} \cdot \vec{\sigma}\right) ^ 2 = I$. Wobec tego, $\hat{v} \cdot \vec{\sigma}$ jest operatorem binarnym.
\end{theorem}

\begin{theorem}
    Spektrum operatora von Neumanna $\hat{v} \cdot \vec{\sigma}$ to zbiór $\{+1, -1\}$. Ponadto

    $$
    \hat{v} \cdot \vec{\sigma} = \Pi_{+} - \Pi_{-},
    $$

    gdzie $\Pi_{\pm} = \frac{1}{2} (I \pm \hat{v} \cdot \vec{\sigma})$ jest projektorem rzutującym na odpowiednią podprzestrzeń własną.
\end{theorem}

\subsubsection{Bazy}

\begin{definition}[Baza Hadamarda]
    $$
        \ket{0'} = \frac{1}{\sqrt{2}} \ket{0} + \frac{1}{\sqrt{2}} \ket{1}
    $$

    $$
        \ket{1'} = \frac{1}{\sqrt{2}} \ket{0} - \frac{1}{\sqrt{2}} \ket{1}
    $$
\end{definition}

\begin{definition}[Baza Bella]
    $$
        \ket{\Phi ^ {+}} = \frac{1}{\sqrt{2}} (\ket{00} + \ket{11}) = \frac{1}{\sqrt{2}} (\ket{0} + \ket{3})
    $$

    $$
        \ket{\Phi ^ {-}} = \frac{1}{\sqrt{2}} (\ket{00} - \ket{11}) = \frac{1}{\sqrt{2}} (\ket{0} - \ket{3})
    $$

    $$
        \ket{\Psi ^ {+}} = \frac{1}{\sqrt{2}} (\ket{01} + \ket{10}) = \frac{1}{\sqrt{2}} (\ket{1} + \ket{2})
    $$

    $$
        \ket{\Psi ^ {-}} = \frac{1}{\sqrt{2}} (\ket{01} - \ket{10}) = \frac{1}{\sqrt{2}} (\ket{1} - \ket{2})
    $$
\end{definition}

\section{Kryteria separowalności}

\section{Wprowadzenie do programowania półokreślonego}

\subsection{Definicje}

\begin{definition}[Operator dodatnio określony]
    Operator liniowy $A$ nazywamy dodatnio określonym, gdy 

    $$
        \forall_{\ket{\psi} \in \mathcal{H}} \bra{\psi} A \ket{\psi} \geq 0.
    $$

    Piszemy wtedy $A \geq 0$.
\end{definition}

\begin{definition}[Programowanie półokreślone]
    Ogólne zagadnienie programowania półokreślonego w postaci pierwotnej definiuje się jako

    $$
        \begin{cases}
            \max \textbf{Tr}[CX] \\
            \text{ze względu na:} \\
            \bullet \textbf{Tr}[A_{i} X] = b_{i}, i = 1, \ldots, p \\
            \bullet X \geq 0
        \end{cases}
    $$

    gdzie

    \begin{itemize}
        \item $X \in M_{n \times n}(\mathbb{R})$ jest macierzą symetryczną traktowaną jako zmienna
        \item $C, A_{i} \in M_{n \times n}(\mathbb{R}), i = 1, \ldots , p$ są danymi macierzami symetrycznymi
        \item $b_{i} \in \mathbb{R}, i = 1, \ldots, p$ są danymi liczbami
    \end{itemize}
\end{definition}

\end{document}
