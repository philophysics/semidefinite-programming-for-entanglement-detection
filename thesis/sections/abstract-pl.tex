\section*{Streszczenie}

W pracy przedstawiono podstawy programowania półokreślonego oraz problematykę separowalności macierzy gęstości, wraz z podaniem niezbędnej wiedzy z algebry liniowej. Praca objęła implementację potrzebnych narzędzi do formułowania warunków separowalności macierzy w oparciu o framework (np. pakiety SeDuMi i YALMIP w środowisku MATLAB/OCTAVE) oraz zilustrowanie zastosowań tych narzędzi na zaproponowanych przykładach.

Wykonane zadania:

\begin{enumerate}
    \item Przegląd literatury odnośnie programowania półokreślonego
    \item Wprowadzenie do metod informatyki kwantowej
    \item Implementacja narzędzia w postaci skryptów w języku MATLAB
    \item Sporządzenie dokumentacji skryptów oraz przykładów
\end{enumerate}

\noindent \textbf{Słowa kluczowe:} informatyka kwantowa

\noindent \textbf{Dziedzina nauki:} Nauki o komputerach i informatyka
