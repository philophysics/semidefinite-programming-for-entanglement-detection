\section{Wprowadzenie do informatyki kwantowej}

\subsection{Przestrzeń Hilberta}

Podamy tutaj definicję przestrzni Hilberta. Najpierw jednak przypomnimy definicję \textit{normy}.

\begin{definition}[Norma]
    Niech $X$ jest przestrzenią liniową nad ciałem $F$ liczb rzeczywistych lub zespolonych. \textit{Normą} nazywamy odwzorowanie $||\cdot||: X \rightarrow [0, + \infty]$ takie, że $\forall a \in F$, $\forall x, y \in X$:

    \begin{enumerate}
        \item $||x + y|| \leq ||x|| + ||y||$
        \item $||a x|| = |a| \cdot ||x||$
        \item $||x|| = 0 \Rightarrow x = 0$
    \end{enumerate}
\end{definition}

\begin{definition}[Przestrzeń Hilberta]
    \textit{Przestrzenią Hilberta} nazywamy przestrzeń liniową wyposażoną w iloczyn skalarny.
\end{definition}

W dalszych rozważaniach będziemy zakładać, że wymiar przestrzni Hilberta jest skończony i wynosi $n$. Element (wektor) tej przestrzeni oznaczamy $\ket{\psi}$, natomiast iloczyn skalarny pomiędzy dwoma wektorami $\ket{\psi}, \ket{\phi} \in \mathcal{H}$ oznaczymy $\bra{\psi}\ket{\phi}$ (taka konwencja określana jest mianem \textit{notacji Diraca}). Przypomnijmy jeszcze, że na przestrzeni Hilberta norma indukowana jest za pomocą iloczynu skalarnego w następujący sposób:

\begin{equation}\label{equation:norm}
    ||\psi|| \equiv \sqrt{\bra{\psi}\ket{\psi}}.
\end{equation}

\subsubsection{Baza kanoniczna}

Jako $\mathcal{B} \equiv \{\ket{i}\}_{i\in I}$ oznaczmy bazę przestrzeni $\mathcal{H}$. Zakładamy też, że $\mathcal{B}$ stanowi ortonormalny układ wektorów -- w przestrzeniach o skończonym wymiarze jesteśmy bowiem w stanie zastosować procedurę Grama-Schmidta. Dla ustalenia uwagi przyjmijmy $I \equiv \{0,1, \ldots , n - 1 \}$ -- przy takich oznaczeniach bazę $\mathcal{B}$ nazywamy bazą \textit{kanoniczną}. Przyjmuje się, że wektory bazy kanonicznej oznaczamy

$$
    \ket{i} \equiv
    \begin{pmatrix}
        0 \\
        \vdots \\
        0 \\
        1 \\
        0 \\
        \vdots \\
        0
    \end{pmatrix},
$$
gdzie nad współczynnikiem $1$ znajduje się $i$ zer.

Dowolny wektor $\ket{\psi} \in \mathcal{H}$ zapiszemy teraz jako

$$
    \ket{\psi} = \sum\limits_{i \in I} \psi_{i}\ket{i}.
$$

Założenie o skończonym wymiarze przestrzeni skutkuje wygodną reprezentacją w bazie $\mathcal{B}$ abstrakcyjnego wektora z przestrzeni Hilberta jako kolumny jego współczynników:

$$
    \ket{\psi} \equiv
    \begin{pmatrix}
        \psi_{0} \\
        \psi_{1} \\
        \vdots \\
        \psi_{n - 1} \\
    \end{pmatrix}.
$$

\subsubsection{Współczynniki wektora w bazie ortonormalnej}
\label{coefficients}

Przypomnimy obecnie wzór na współczynniki wektora w rozwinięciu w bazie $\mathcal{B}$, która zgodnie z założeniem stanowi układ ortonormalny, tzn. 

$$
    \bra{i}\ket{j} = \delta_{ij},
$$
gdzie dwuargumentowa funkcja $\delta_{ij}$, zwana \textit{deltą Kroneckera}:

$$
    \delta_{ij} \equiv
    \begin{cases}
        1, \text{dla } i = j \\
        0, \text{dla } i \neq j
    \end{cases}
$$

Niech więc dowolnie ustalony wektor bazowy $\ket{\alpha}$ zostanie przemnożony skalarnie przez wektor $\ket{\psi}$. Pamiętając o własnościach iloczynu skalarnego (liniowość w drugim składniku) napiszemy 

$$
    \bra{\alpha}\ket{\psi} = \bra{\alpha}\left(\sum\limits_{i}\psi_{i}\ket{i}\right) = \sum\limits_{i}\psi_{i} \bra{\alpha}\ket{i} = \sum\limits_{i}\psi_{i}\delta_{\alpha i}=\psi_{\alpha}
$$
tzn.

$$
    \psi_{\alpha}=\bra{\alpha}\ket{\psi}.
$$

\subsubsection{Iloczyn skalarny -- równoważne podejście}

Iloczyn skalarny wektorów $\ket{\psi}, \ket{\phi} \in \mathcal{H}$ można wyznaczyć wprost z definicji funkcji $\bra{\psi}\ket{\phi}$, a także za pomocą sumy współczynników tych wektorów. Dla

$$
    \ket{\psi} = \sum\limits_{i\in I}\psi_{i}\ket{i}
$$
oraz

$$
    \ket{\phi} = \sum\limits_{i\in I}\phi_{i}\ket{i}
$$
mamy bowiem

$$
    \bra{\psi}\ket{\phi} = \left(\sum\limits_{i \in I}\psi_{i}\ket{i}\right)^{*}\left(\sum\limits_{j\in I}\phi_{j}\ket{j}\right) = \sum\limits_{i, j \in I}\psi_{i} ^ {*}\phi_{j}\bra{i}\ket{j}=\sum\limits_{i \in I}\psi_{i}^{*}\phi_{i}
$$

\subsubsection{Przestrzeń sprzężona}

Samo wyrażenie $\bra{\psi}\ket{\phi}$ także w konwencji Diraca traktować można dwojako -- formalnie bowiem w drugim składniku iloczynu skalarnego występują wektory postaci $\ket{\cdot}$ rozpatrywanej przestrzeni Hilberta $\mathcal{H}$, nazywane \textit{ketami}, natomiast w pierwszym -- wektory zapisywane jako $\bra{\cdot}$, nazywane \textit{bra} (oba te pojęcia pochodzą od słowa \textit{bracket}). Wektory \textit{bra} definiuje się ściśle jako elementy \textit{przestrzeni sprzężonej} do przestrzeni $\mathcal{H}$.

\begin{definition}
    Przestrzenią sprzężoną do przestrzeni Hilberta $\mathcal{H}$ nazywamy przestrzeń $\mathcal{H}^{*}$ liniowych funkcjonałów $\bra{\cdot} : \mathcal{H} \rightarrow \mathbb{C}$.
\end{definition}

\begin{remark} 
    Funkcjonały z przestrzeni sprzężonej są liniowe, ponieważ iloczyn skalarny jest liniowy w drugim argumencie.
\end{remark}

Okazuje się, że w przestrzeniach Hilberta o wymiarze skończonym występuje wzajemna jednoznaczność bra i ketów, tzn. wektor $\ket{\psi} \in \mathcal{H}$ można rozpatrywać równoważnie jako wektor $\bra{\psi} \in \mathcal{H}^{*}$ powstały z $\ket{\psi}$ na drodze przekształcenia, które w przestrzeniach o skończonym wymiarze wygląda następująco: z reprezentacji wektora jako kolumny współczynników i z formuły na iloczyn skalarny w bazie ortonormalnej mamy, że dla danego keta

$$
    \ket{\psi} =
    \begin{pmatrix}
        \psi_{0} \\
        \psi_{1} \\
        \vdots \\
        \psi_{n - 1} \\
    \end{pmatrix}
$$
odpowiadający mu funkcjonał $\bra{\psi} \in \mathcal{H}^{*}$ można zapisać jako

$$
    \bra{\psi} = \begin{pmatrix} \psi_{0}^{*} & \psi_{1}^{*} & \ldots & \psi_{n - 1}^{*} \end{pmatrix}.
$$

Wobec tego, bra powstaje z keta na skutek transpozycji i sprzężenia zespolonego. Teraz iloczyn skalarny $\bra{\psi}\ket{\phi}$ wektorów $\ket{\psi}, \ket{\phi} \in \mathcal{H}$ uzyskuje dodatkowy sens zwykłego mnożenia wiersza przez kolumnę.

\subsubsection{Podstawowe bazy}

Podamy definicje podstawowych baz pojawiających się w kwantowej informatyce.

\begin{definition}[Baza Hadamarda]
    $$
        \ket{0'} = \frac{1}{\sqrt{2}} \ket{0} + \frac{1}{\sqrt{2}} \ket{1}
    $$

    $$
        \ket{1'} = \frac{1}{\sqrt{2}} \ket{0} - \frac{1}{\sqrt{2}} \ket{1}
    $$
\end{definition}

\begin{definition}[Baza Bella]
    $$
        \ket{\Phi ^ {+}} = \frac{1}{\sqrt{2}} (\ket{00} + \ket{11}) = \frac{1}{\sqrt{2}} (\ket{0} + \ket{3})
    $$

    $$
        \ket{\Phi ^ {-}} = \frac{1}{\sqrt{2}} (\ket{00} - \ket{11}) = \frac{1}{\sqrt{2}} (\ket{0} - \ket{3})
    $$

    $$
        \ket{\Psi ^ {+}} = \frac{1}{\sqrt{2}} (\ket{01} + \ket{10}) = \frac{1}{\sqrt{2}} (\ket{1} + \ket{2})
    $$

    $$
        \ket{\Psi ^ {-}} = \frac{1}{\sqrt{2}} (\ket{01} - \ket{10}) = \frac{1}{\sqrt{2}} (\ket{1} - \ket{2})
    $$
\end{definition}

\subsection{Przestrzeń operatorów}

\subsubsection{Elementy macierzowe, ślad, komutator}

Przedstawimy podstawowe definicje i własności operatorów.

\begin{definition}
    \label{definition:matrix-element}
    Elementem macierzowym w bazie $\mathcal{B}$ operatora liniowego $A$ nazywamy liczbę

    $$
        A_{ij} \equiv \bra{i} A \ket{j},
    $$
    gdzie $\ket{i}, \ket{j} \in \mathcal{B}$. Ponadto, elementy macierzowe postaci $A_{ii}$ nazywamy elementami diagonalnymi operatora $A$.
\end{definition}

Operator $A$ można zapisać w następujący sposób:

$$
    A = \sum\limits_{i, j \in I} A_{ij} \ket{i}\bra{j}.
$$

\begin{remark}
    Zapis postaci $\bra{\cdot}\cdot\ket{\cdot}$ stosujemy w celach mnemotechnicznych -- obowiązuje tutaj ogólna konwencja notacyjna:

    \begin{itemize}
        \item $\ket{A \psi} \equiv A \ket{\psi}$
        \item $\bra{A \psi} \equiv \bra{\psi} A ^ {\dag}$
    \end{itemize}
\end{remark}

\begin{definition}[Ślad operatora]
    Śladem operatora liniowego $A$ nazywamy sumę jego elementów diagonalnych:

    $$
        \textbf{Tr}[A] \equiv \sum\limits_{i \in I} A_{ii}.
    $$
\end{definition}

\begin{theorem}[Ślad operatora jest cykliczny]
    Zachodzi

    $$
        \textbf{Tr}[AB] = \textbf{Tr}[BA].
    $$
\end{theorem}

\begin{proof}
    Niech $A \in M_{n \cross m}(\mathbb{C})$, $B \in M_{m \cross n}(\mathbb{C})$. Wówczas

    $$
        \textbf{Tr}[AB] = \sum\limits_{i = 1}^{n} (AB)_{ii} = \sum\limits_{i = 1}^{n} \sum\limits_{j = 1} ^ {m} A_{ij} B_{ji} = \sum\limits_{j = 1} ^{m} \sum\limits_{i = 1} ^{n} B_{ji} A_{ij} = \sum\limits_{j = 1} ^ {m} (BA)_{jj} = \textbf{Tr}[BA].
    $$
\end{proof}

Wprowadzimy teraz pojęcie \textit{komutatora} operatorów liniowych.

\begin{definition}[Komutator]
    Dla operatorów liniowych $A$ oraz $B$ działających na przestrzeni Hilberta $\mathcal{H}$ ich komutator definiujemy jako operator liniowy

    $$
        [A, B] \equiv A \circ B - B \circ A,
    $$
    gdzie działanie $\circ$ oznacza złożenie operatorów (w przypadku skończenie wymiarowym i reprezentacji macierzowej operatora jest to mnożenie macierzy). Ponadto, jeżeli $[A, B] = \hat{0}$ (gdzie $\hat{0}$ jest operatorem zerowym) to mówimy, że operatory $A$ i $B$ komutują.

\end{definition}

\subsubsection{Operatory hermitowskie}

Wprowadzimy pojęcie operatora hermitowskiego i podamy jego podstawowe własności.

\begin{definition}[Sprzężenie hermitowskie]
    Niech $A$ jest operatorem liniowym z $\mathcal{H}$ do $\mathcal{H}$. Operatorem sprzężonym hermitowsko do $A$ nazywamy taki operator $A ^ {\dag}$, że dla każdych $\ket{\psi}, \ket{\phi} \in \mathcal{H}$ zachodzi

    $$
        \bra{\psi}A\ket{\phi} = \bra{\phi}A^{\dag}\ket{\psi}^{*}.
    $$
    W przypadku przestrzeni o skończonym wymiarze definicja ta redukuje się do postaci

    $$
        A ^ {\dag} \equiv (A ^ {*}) ^ T = (A ^ {T}) ^ {*}.
    $$
\end{definition}

\begin{definition}[Operator hermitowski]
    Operator liniowy $A$ nazywamy hermitowskim (samosprzężonym), gdy

    $$
        A = A ^ {\dag}
    $$
\end{definition}

\begin{theorem}
    Wartości własne operatora hermitowskiego są rzeczywiste.
\end{theorem}

\begin{proof}
    Niech $A$ jest operatorem hermitowskim i niech $\ket{\lambda}$ jest wektorem własnym odpowiadającym wartości własnej $\lambda$. Ponieważ $A = A ^ {\dag}$, to

    $$
        \bra{\lambda} A \ket{\lambda} = \bra{\lambda} A ^ {\dag} \ket{\lambda} ^ {*} = \bra{\lambda} A \ket{\lambda} ^ {*}.
    $$
    Skoro $\bra{\lambda} A \ket{\lambda} = \lambda \bra{\lambda}\ket{\lambda}$, to $\lambda \bra{\lambda}\ket{\lambda} = (\lambda \bra{\lambda}\ket{\lambda}) ^ {*} = \lambda ^ {*} \bra{\lambda}\ket{\lambda}$. Wobec tego $\lambda = \lambda ^ {*}$. 
\end{proof}

\subsubsection{Operatory dodatnio określone}

Wprowadzimy pojęcie i podamy podstawowe własności operatorów dodatnio określonych.

\begin{definition}[Operator dodatnio określony]
    \label{definition:semidefinite}
    Operator liniowy $A$ nazywamy dodatnio określonym, gdy

    $$
        \forall_{\ket{\psi} \in \mathcal{H}} \bra{\psi} A \ket{\psi} \geq 0.
    $$

    Piszemy wtedy $A \geq 0$.
\end{definition}

\begin{theorem}
    \label{theorem:positive}
    Operator dodatnio określony jest hermitowski.
\end{theorem}

\begin{theorem}
    \label{theorem:positivevalues}
    Wartości własne operatora dodatnio określonego są nieujemnymi liczbami rzeczywistymi.
\end{theorem}

\begin{theorem}
    \label{theorem:adaga}
    Jeżeli $A$ jest operatorem liniowym, to operator $A ^ {\dag} A$ jest dodatnio określony.
\end{theorem}

\begin{proof}
    Niech $\ket{\psi} \in \mathcal{H}$ będzie dowolnym wektorem. Wtedy z równania (\ref{equation:norm}) mamy

    $$
        \bra{\psi} A ^ {\dag} A \ket{\psi} = \bra{A \psi}\ket{A \psi}=\left\|A \ket{\psi}\right\|^2 \geq 0,
    $$
    więc $A ^ {\dag} A$ jest operatorem dodatnio określonym.
\end{proof}

\begin{theorem}
    Jeżeli $A$, $B$ są operatorami na $\mathcal{H}$ dodatnio określonymi, to ich suma $A + B$ jest operatorem dodatnio określonym.
\end{theorem}

\begin{proof}
    Niech $\ket{\psi} \in \mathcal{H}$. Wtedy

    $$
        \bra{\psi} (A + B) \ket{\psi}=\bra{\psi} A \ket{\psi} + \bra{\psi} B \ket{\psi} \geq 0.
    $$
\end{proof}

Jako definicję operatora dodatnio określonego równoważną definicji \ref{definition:semidefinite} można przyjąc poniższą definicję \textit{operatora dodatniego} (nazwa została przyjęta jedynie po to, aby odróżnić od siebie dokładne treści obu definicji):

\begin{definition}[Operator dodatni]
    Operator liniowy $A$ nazywamy dodatnim, gdy jest hermitowski i ma nieujemne spektrum.
\end{definition}

Okazuje się bowiem, że z twierdzeń \ref{theorem:positive} oraz \ref{theorem:positivevalues}  wynika

\begin{corollary}
    \label{corollary:iff}
    Operator liniowy $A$ jest dodatnio określony $\Longleftrightarrow$ jest dodatni.
\end{corollary}

\subsubsection{Przestrzeń Hilberta--Schmidta}

Dla danej przestrzeni Hilberta $\mathcal{H}$ z bazą $\mathcal{B}$ wprowadzamy przestrzeń operatorów liniowych $A: \mathcal{H} \rightarrow \mathcal{H}$ z działaniem standardowego dodawania dwóch operatorów i działaniem mnożenia operatora przez liczbę zespoloną. Przestrzeń taka jest przestrzenią liniową. Wyposażamy ją w działanie dwuargumentowe postaci

$$
    \bra{A}\ket{B} \equiv \textbf{Tr}[A ^ {\dag} B].
$$

Łatwo pokazać, że tak zdefiniowana funkcja jest iloczynem skalarnym (nazywamy ją iloczynem skalarnym Hilberta--Schmidta) -- przestrzeń operatorów rozszerzyliśmy wobec tego do przestrzeni Hilberta. Nazywamy ją przestrzenią Hilberta-Schmidta sprzężoną z przestrzenią $\mathcal{H}$ i oznaczamy $\mathcal{HS}$.


\subsection{Iloczyn tensorowy}

\subsubsection{Definicja ogólna}

W niniejszym paragrafie postaramy się przedstawić w możliwie najogólniejszy sposób definicję iloczynu tensorowego, podając następnie praktyczną jego realizację w przestrzeniach o wymiarze skończonym.

Weźmy dwie przestrzenie Hilberta $\mathcal{H}_{A}$ i $\mathcal{H}_{B}$. Dla wektorów $\ket{A} \in \mathcal{H}_{A}$ oraz $\ket{B} \in \mathcal{H}_{B}$ wektor będący ich \textit{iloczynem tensorowym} (oznaczany $\ket{A} \otimes \ket{B}$) formalnie jest elementem przestrzeni będącej \textit{iloczynem tensorowym przestrzeni} $\mathcal{H}_{A}$ i $\mathcal{H}_{B}$. Najpierw wprowadzimy więc pojęcie iloczynu tensorowego przestrzeni, a następnie -- definicję działania mnożenia tensorowego wektorów.

\begin{definition}[Iloczyn tensorowy przestrzeni]
    Iloczynem tensorowym przestrzeni Hilberta $\mathcal{H}_A$ (z ortonormalną bazą $\ket{i^{A}}$) i $\mathcal{H}_B$ (z ortonormalną bazą $\ket{j^{B}}$) nazywamy przestrzeń

    $$
        \mathcal{H}_{AB} \equiv \mathcal{H}_A \otimes \mathcal{H}_B,
    $$
    której elementy stanowią wszystkie wektory postaci

    \begin{equation}
        \label{equation:product}
        \ket{C} \equiv \ket{A} \otimes \ket{B}, \text{dla } \ket{A} \in \mathcal{H}_A, \ket{B} \in \mathcal{H}_B.
    \end{equation}

    Dodatkowo, wektory tej przestrzeni stanowią z definicji wszystkie kombinacje liniowe układu wektorów $\left|i^{A}\right\rangle\otimes\left|j^{B}\right\rangle$, tzn. zbiór wektorów postaci

    \begin{equation}
        \label{equation:entangled}
        \ket{C} = \sum\limits_{ij} c_{ij} \ket{i ^ {A}} \otimes \ket{j ^ {B}}.
    \end{equation}
\end{definition}

\begin{definition}[Iloczyn tensorowy wektorów]
    Dla danych przestrzeni Hilberta $\mathcal{H}_A$ oraz $\mathcal{H}_B$ działanie mnożenia tensorowego wektorów z $\mathcal{H}_A$ z wektorami z $\mathcal{H}_B$ definiuje się jako funkcję $\otimes: \mathcal{H}_A \times \mathcal{H}_B \rightarrow \mathcal{H}_{AB}$, mającą własność liniowości w obu swoich składnikach, tzn.

    \begin{enumerate}
        \item \textbf{Jednorodność w obu składnikach.}
            Dla każdego skalara $z \in \mathbb{C}$ i dla każdych wektorów $A \in \mathcal{H}_A$ oraz $B \in \mathcal{H}_B$

            $$
                z (\ket{A} \otimes \ket{B}) = (z \ket{A}) \otimes \ket{B} = \ket{A} \otimes (z \ket{B})
            $$
        \item \textbf{Addytywność w pierwszym składniku.}
            Dla każdych wektorów $\ket{A_1}, \ket{A_2} \in \mathcal {H}_A$ oraz $\ket{B} \in \mathcal{H}_B$

            $$
                (\ket{A_1} + \ket{A_2}) \otimes \ket{B} = \ket{A_1} \otimes \ket{B} + \ket{A_2} \otimes \ket{B}
            $$
        \item \textbf{Addytywność w drugim składniku.}
            Dla każdych wektorów $\ket{A} \in \mathcal{H}_A$ oraz $\ket{B_1}, \ket{B_2} \in \mathcal{H}_B$

            $$
                \ket{A} \otimes (\ket{B_1} + \ket{B_2}) = \ket{A} \otimes \ket{B_1} + \ket{A} \otimes \ket{B_2}
            $$
    \end{enumerate}
\end{definition}

\begin{remark}[Oznaczenia]
    Dla $\ket{A} \in \mathcal{H}_{A}$ i $\ket{B} \in \mathcal{H}_{B}$ oznaczać będziemy

    $$
        \ket{A} \otimes \ket{B} \equiv \ket{A} \ket{B}.
    $$

    Czasem piszemy nawet $\ket{A} \otimes \ket{B} \equiv \ket{A B}$.
\end{remark}

\begin{remark}
    \label{remark:classes}
    Z liniowości iloczynu tensorowego wynika, że każdy wektor z $\mathcal{H}_{AB}$ postaci (\ref{equation:product}) można zapisać w postaci (\ref{equation:entangled}). Dla

    $$
        \ket{A} = \sum\limits_{i} c_i ^ {(A)} \ket{i^{(A)}}
    $$
    oraz

    $$
        \ket{B} = \sum\limits_{j} c_j ^ {(B)} \ket{j^{(B)}}
    $$
    mamy bowiem

    $$
        \ket{A} \ket{B} = \left(\sum\limits_{i} c_i ^ {(A)} \ket{i^{(A)}}\right) \left(\sum\limits_{j} c_j ^ {(B)} \ket{j^{(B)}}\right)
    $$

    $$
        = \sum\limits_{ij} c_{i} ^ {(A)} c_{j} ^ {(B)} \ket{i ^ {(A)}} \ket{j ^ {(B)}} = \sum\limits_{ij} c_{ij} \ket{i ^ {(A)}} \ket{j ^ {(B)}},
    $$
    jeżeli tylko $c_{ij} \equiv c_{i} ^ {(A)} c_{j} ^ {(B)}$.

    Implikacja odwrotna jednak nie zachodzi: nie każdy wektor dany w postaci (\ref{equation:entangled}) da się zapisać w postaci (\ref{equation:product}) (podamy później przykład takiego wektora). Ta własność zdefiniowanej przez nas struktury iloczynu tensorowego będzie miała wielkie znaczenie przy wprowadzaniu pojęcia \textit{splątania}.
\end{remark}

\begin{remark}
    \label{remark:product-vector}
    O tych wektorach przestrzeni $\mathcal{H}_{AB}$, które można zapisać w postaci (\ref{equation:product}) mówimy, że są wektorami produktowymi.
\end{remark}

\subsubsection{Iloczyn tensorowy w przestrzeniach o skończonym wymiarze}

Podamy teraz definicję iloczynu tensorowego w przestrzeniach Hilberta o wymiarze skończonym określonych nad $\mathbb{C}$.

Wprowadzimy najpierw pewne działanie wykonywane na macierzach -- rozważane ogólnie nie ma ono nic wspólnego z teorią kwantów.

\begin{definition}[Iloczyn Kroneckera macierzy]
    Dla macierzy

    $$
        A \in M_{p \times q}(\mathbb{C})
    $$
    i

    $$
        B \in M_{r\times s}(\mathbb{C})
    $$
    ich iloczyn Kroneckera zdefiniowany jest jako macierz

    $$
        C \in M_{pr\times qs}(\mathbb{C})
    $$
    dana wzorem

    \begin{equation}
        \label{equation:kronecker-product}
        C \equiv A \otimes B =
        \begin{pmatrix}
            A_{00} B & A_{01} B & \ldots & A_{0, q - 1} B \\
            A_{10} B & A_{11} B & \ldots & A_{1, q - 1} B \\
            \vdots & \vdots & \vdots & \vdots \\
            A_{p - 1, 0} B & A_{p - 1, 1} B & \ldots & A_{p - 1, q - 1} B
        \end{pmatrix}.
    \end{equation}
\end{definition}

\begin{example}[Iloczyn Kroneckera]
    Dla danych macierzy

    $$
        A =
        \begin{pmatrix}
            a & b \\
            c & d
        \end{pmatrix}
    $$

    $$
        B =
        \begin{pmatrix}
            e & f & g \\
            h & i & j
        \end{pmatrix}
    $$
    ich iloczyn Kroneckera przyjmuje wartość

    $$
        C =
        \begin{pmatrix}
            a B & b B \\
            c B & d B
        \end{pmatrix}
        =
        \begin{pmatrix}
            a e & a f & a g & b e & b f & b g \\
            a h & a i & a j & b h & b i & b j \\
            c e & c f & c g & d e & d f & d g \\
            c h & c i & c j & d h & d i & d j
        \end{pmatrix}
    $$
\end{example}

Podamy podstawowe własności iloczynu Kroneckera.

\begin{theorem}
    Jeżeli $A$, $B$, $C$, $D$ są macierzami takimi, że wyrażenia $A C$ i $B D$ mają sens jako mnożenie algebraiczne macierzy, to

    \begin{equation}
        (A \otimes B) (C \otimes D) = (A C) \otimes (B D),
    \end{equation}
    gdzie $\otimes$ jest iloczynem Kroneckera macierzy.
\end{theorem}

Można też udowodnić twierdzenie ogólniejsze:

\begin{theorem}
    \label{theorem:generic}
    Niech $\ket{a_{1}}, \ket{a_{2}} \in \mathcal{H}_{A}$, $\ket{b_{1}}, \ket{b_{2}} \in \mathcal{H}_{B}$. Wówczas

    \begin{equation}
        (\ket{a_{1}} \ket{b_{1}}) (\bra{a_{2}} \bra{b_{2}}) = \ket{a_{1}} \bra{a_{2}} \otimes \ket{b_{1}} \bra{b_{2}}.
    \end{equation}
\end{theorem}

Ponadto mamy

\begin{theorem}
    Jeżeli $A$, $B$ są macierzami, to

    \begin{equation}
        (A \otimes B) ^ {\dag} = A ^ \dag \otimes B ^ \dag.
    \end{equation}
\end{theorem}

\begin{corollary}
    Iloczyn Kroneckera zachowuje hermitowskość, tzn. jeżeli macierze $A$ i $B$ są hermitowskie, to macierz $A \otimes B$ również.
\end{corollary}

Można też pokazać, że w odniesieniu do operatora \textbf{Tr} mamy następujące

\begin{theorem}
    Jeżeli $A$ i $B$ są dowolnymi macierzami kwadratowymi, to

    \begin{equation}
        \textbf{Tr}[A \otimes B] = \textbf{Tr}[A] \cdot \textbf{Tr}[B].
    \end{equation}
\end{theorem}

Należy podkreślić, że działanie (\ref{equation:kronecker-product}) \textit{spełnia własności iloczynu tensorowego}. Pokażemy później, że w kwantowej teorii informacji interesującymi nas przestrzeniami Hilberta są zwykłe $n$-wymiarowe przestrzenie kolumn liczb zespolonych $V ^ n (\mathbb{C})$ i sprzężone z nimi przestrzenie Hilberta--Schmidta, tzn. przestrzenie macierzy kwadratowych wymiaru $n$: $M_{n\times n}(\mathbb{C})$. Naturalnym rozwiązaniem jest przyjęcie w tych przestrzeniach iloczynu tensorowego według reguły (\ref{equation:kronecker-product}). Wobec tego, w kwantowej teorii informacji iloczyn tensorowy oznacza po prostu iloczyn Kroneckera.

Uogólnimy obecnie rozważania podane w definicji \ref{definition:matrix-element}.

Niech dane są skończenie wymiarowe przestrzenie Hilberta $\mathcal{H}_{A}$ oraz $\mathcal{H}_{B}$ z bazami odpowiednio

$$
    \{\ket{a_{i}}\}_{i \in [n_{A}]}
$$
oraz

$$
    \{\ket{b_{j}}\}_{j \in [n_{B}]}.
$$

Niech dany jest też operator liniowy $A$ działający na przestrzeni $\mathcal{H}_{A} \otimes \mathcal{H}_{B}$. Dla

$$
    i, k \in [n_{A}],
$$

$$
    j,l \in [n_{B}]
$$
elementem macierzowym tego operatora w bazie $\{\ket{a_{i}} \ket{b_{j}}\}_{i \in [n_{A}], j \in [n_{B}]}$ przestrzeni złożonej nazywamy liczbę

$$
    A_{\stackrel{ij}{kl}} \equiv \bra{a_{i}} \bra{b_{j}} A \ket{a_{k}} \ket{b_{l}}.
$$

Operator $A$ zapiszemy wtedy w rozważanej bazie jako

$$
    A = \sum\limits_{i, k \in [n_{A}]; j, l \in [n_{B}]} A_{\stackrel{ij}{kl}} (\ket{a_{i}} \ket{b_{j}}) (\bra{a_{k}} \bra{b_{l}})
$$

\begin{equation}
    \label{equation:elements}
    \stackrel{Tw. \ref{theorem:generic}}{=} \sum\limits_{i, k \in [n_{A}]; j, l \in [n_{B}]} A_{\stackrel{ij}{kl}} \ket{a_{i}} \bra{a_{k}} \otimes \ket{b_{j}} \bra{b_{l}}.
\end{equation}

\subsection{Twierdzenie spektralne}

\subsubsection{Projektory}

\begin{definition}
    Niech $\mathcal{H}_{A}$, $\mathcal{H}_{B}$ będą skończenie wymiarowymi przestrzeniami Hilberta. Dla $\ket{A} \in \mathcal{H}_A$ i $\ket{B} \in \mathcal{H}_B$ wyrażenie $\ket{A} \bra{B}$ wyznaczone według reguły (\ref{equation:kronecker-product}) i będące wektorem z przestrzeni $\mathcal{H}_A \otimes \mathcal{H}_{B} ^ {*}$, gdzie $\mathcal{H}_{B} ^ {*}$ jest przestrzenią sprzężoną do przestrzeni $\mathcal{H}_B$, nazywamy iloczynem zewnętrznym wektorów $\ket{A}$ i $\ket{B}$.
\end{definition}

Niech dana jest przestrzeń Hilberta $\mathcal{H}$ wymiaru $n$ z bazą ortonormalną $\{\ket{i}\}_{i = 0} ^ {n - 1}$. Wtedy (zob. \ref{coefficients}) dowolny wektor $\ket{\psi} \in \mathcal{H}$ można zapisać jako

$$
    \ket{\psi} = \sum\limits_{i = 0} ^ {n - 1} \ket{i} \underbrace{\bra{i}\ket{\psi}}_{\psi_{i}}.
$$

Każdemu wektorowi $\ket{i}$ z bazy przypiszmy teraz operator liniowy $\ket{i} \bra{i} : \mathcal{H} \rightarrow \mathcal{H}$, którego działanie zdefiniujemy następująco:

$$
    (\ket{i} \bra{i}) \ket{\psi} \equiv \ket{i} \bra{i}\ket{\psi}.
$$

Wtedy

$$
    \ket{\psi} = \left(\sum_{i = 0} ^ {n - 1} \ket{i} \bra{i} \right) \ket{\psi},
$$
tzn.

\begin{equation}
    \label{equation:identity}
    \sum\limits_{i = 0} ^ {n - 1} \ket{i} \bra{i} = I.
\end{equation}

Oznaczmy

$$
    \Pi_{i} \equiv \ket{i} \bra{i}, i = 0, 1, \ldots, n - 1.
$$

\begin{definition}
    Niech dana jest przestrzeń liniowa i jej podprzestrzeń $V$. Niech bazą podprzestrzeni $V$ jest $\mathcal{B}_{V}$. Projektorem rzutującym na podprzestrzeń $V$ nazywamy operator

    $$
        \Pi_V \equiv \sum\limits_{\ket{i} \in \mathcal{B}_V} \ket{i} \bra{i}.
    $$

    Ponadto, jeżeli baza $\mathcal{B}_V$ jest układem ortonormalnym, to projektor $\Pi_V$ nazywamy ortonormalnym.
\end{definition}

Własność (\ref{equation:identity}) układu ortonormalnych projektorów $\{\ket{i} \bra{i}\}_{i = 0} ^ {n - 1}$ nazywamy \textit{rozkładem identyczności}. Ponadto, złożenie dwóch projektorów z tego układu spełnia

$$
    \Pi_{i} \Pi_{j} = \delta_{ij} \Pi_{i}.
$$

Mamy bowiem dla dowolnego wektora $\ket{\psi} \in \mathcal{H}$

$$
    \Pi_{i} \Pi_{j} \ket{\psi} = \Pi_{i} \ket{j} \bra{j}\ket{\psi} = \ket{i} \bra{i}\ket{j} \bra{j}\ket{\psi} = \delta_{ij} \Pi_{i} \ket{\psi}.
$$

Latwo też pokazać, że konstrukcja skończenie wymiarowych projektorów ortonormalnych za pomocą iloczynu Kroneckera implikuje ich hermitowskość.

\begin{fact}
    Projektory ortonormalne są operatorami dodatnio określonymi.
\end{fact}

\begin{proof}
    Jeżeli $\Pi$ jest projektorem ortonormalnym, to

    $$
        \Pi ^ {\dag} \Pi = \Pi ^ 2 =\Pi.
    $$

    Z drugiej strony, z twierdzenia \ref{theorem:adaga} wynika, że operator $\Pi ^ {\dag} \Pi$ jest dodatnio określony.
\end{proof}

\subsubsection{Rozkład spektralny}

Przedstawimy tutaj ideę \textit{rozkładu spektralnego} operatora. Samo pojęcie zdefiniujemy rozważając przypadek, gdy działa on w pewnej przestrzeni Hilberta $\mathcal{H}$. Konstrukcję uogólnimy następnie do przestrzeni $\mathcal{H}_{AB} \equiv \mathcal{H}_A \otimes \mathcal{H}_B$.

\paragraph{Przestrzeń $\mathcal{H}$}

Rozważmy operator liniowy $A$ działający w przestrzeni Hilberta $\mathcal{H}$.

\begin{definition}
    Mówimy, że operator liniowy $A$ \textit{ma rozkład spektralny}, gdy da się go przedstawić w postaci

    $$
        A = \sum\limits_{\lambda \in \sigma(A)} \lambda \Pi_{\lambda},
    $$
    gdzie $\sigma(A)$ jest zbiorem różnych wartości własnych operatora $A$, natomiast $\Pi_{\lambda}$ jest projektorem ortonormalnym rzutującym na podprzestrzeń własną odpowiadającą wartości własnej $\lambda$.
\end{definition}

\begin{definition}[Operator normalny]
    Operator liniowy $A$ nazywamy normalnym gdy

    $$
        A ^ {\dag} A = A A ^ {\dag}
    $$
\end{definition}

\begin{theorem}[Twierdzenie Spektralne]
    \label{theorem:spectral}
    Operator liniowy ma rozkład spektralny $\Leftrightarrow$ jest normalny.
\end{theorem}

Ponieważ operator hermitowski jest normalny:

$$
    A = A ^ {\dag} \Longrightarrow A ^ {\dag} A = A A ^ {\dag} = A ^ 2,
$$
to z twierdzenia \ref{theorem:spectral} mamy

\begin{corollary}
    \label{corollary:hermitian}
    Operator hermitowski ma rozkład spektralny.
\end{corollary}

Zapiszmy jeszcze prosty fakt.

\begin{fact}
    Jeżeli operatory liniowe $A$ oraz $B$ działające na przestrzeni Hilberta $\mathcal{H}$ wymiaru $n$ posiadają rozkład spektralny we wspólnej bazie ortonormalnych projektorów $\{\Pi_{i}\}_{i = 0} ^ {n - 1}$ postaci

    $$
        A = \sum\limits_{i = 0} ^ {n - 1} a_{i} \Pi_{i},
    $$

    $$
        B = \sum\limits_{i = 0} ^ {n - 1} b_{i} \Pi_{i},
    $$
    to operatory $A$ i $B$ komutują.
\end{fact}

\begin{proof}
    Zachodzi

    $$
        A B = \left(\sum\limits_{i = 0} ^ {n - 1} a_{i} \Pi_{i}\right) \left(\sum\limits_{j = 0} ^ {n - 1} b_{j} \Pi_{j}\right) = \sum\limits_{i, j = 0} ^ {n - 1} a_{i} b_{j} \Pi_{i} \Pi_{j} = \sum\limits_{i = 0} ^ {n - 1} a_{i} b_{i} \Pi_{i}.
    $$

    Podobnie

    $$
        B A = \sum\limits_{i = 0} ^ {n - 1} b_{i} a_{i} \Pi_{i}.
    $$

    Wobec tego $A B = B A$.
\end{proof}

\paragraph{Przestrzeń $\mathcal{H}_{AB}$}

Rozważmy teraz przestrzenie Hilberta $\mathcal{H}_{A}$ oraz $\mathcal{H}_{B}$, odpowiednio wymiaru $n_{A}$ i $n_{B}$ i z bazami ortonormalnych projektorów

$$
    \left\{\Pi_{i} ^ {(A)}\right\}_{i = 0} ^ {n_{A} - 1},
$$

$$
    \left\{\Pi_{j} ^ {(B)}\right\}_{j = 0} ^ {n_{B} - 1}.
$$

Uogólnienia dokonamy dla operatorów produktowych działających w $\mathcal{H}_{AB}$, tzn. dla takich operatorów $E ^ {(AB)}$, które w przestrzeni Hilberta--Schmidta $\mathcal{HS}_{AB} \equiv \mathcal{HS}_A \otimes \mathcal{HS}_B$ sprzężonej z przestrzenią Hilberta $\mathcal{H}_{AB}$, są wektorami produktowymi (\ref{equation:product}), tzn.

$$
    E ^ {(AB)} = E ^ {(A)} \otimes E ^ {(A)},
$$
gdzie pewne operatory $E ^ {(A)}$, $E ^ {(B)}$ działające na przestrzeni $\mathcal{H}_A$ oraz $\mathcal{H}_B$ same mają rozkłady spektralne

$$
    E ^ {(A)} = \sum\limits_{i} \lambda_{i} ^ {(A)} \Pi_{i} ^ {(A)},
$$

$$
    E ^ {(B)} = \sum\limits_{j} \lambda_{j} ^ {(B)} \Pi_{j} ^ {(B)}.
$$

Wtedy

$$
    E ^ {(AB)} = \left(\sum\limits_{i} \lambda_{i} ^ {(A)} \Pi_{i} ^ {(A)}\right) \otimes \left(\sum\limits_{j} \lambda_{j} ^ {(B)} \Pi_{j} ^ {(B)}\right)
$$

$$
    = \sum\limits_{ij} \lambda_{i} ^ {(A)} \lambda_{j} ^ {(B)} \Pi_{i} ^ {(A)} \otimes \Pi_{j} ^ {(B)}.
$$

Definiując $\lambda_{ij} ^ {(AB)} \equiv \lambda_{i} ^ {(A)} \lambda_{j} ^ {(B)}$, $\Pi_{ij} ^ {(AB)} \equiv \Pi_{i} ^ {(A)} \otimes \Pi_{j} ^ {(B)}$, otrzymujemy rozkład spektralny operatora $E ^ {(AB)}$ na projektory działające w przestrzeni $\mathcal{H}_{AB}$ postaci

$$
    E ^ {(AB)} = \sum\limits_{ij} \lambda_{ij} ^ {(AB)} \Pi_{ij} ^ {(AB)},
$$
gdzie bazą ortonormalnych projektorów jest zbiór

$$
    \left\{\Pi_{ij} ^ {(AB)}\right\}_{i \in [n_{A}], j \in [n_{B}]},
$$
zauważmy jednak, że niekoniecznie wszystkie tak określone wartości własne są różne.

\subsection{Operatory binarne i operatory von Neumanna}

Opiszemy tutaj pewną klasę operatorów hermitowskich.

\subsubsection{Operatory binarne}

\begin{definition}[Operator binarny]
    Operator liniowy $A$ nazywamy binarnym, gdy $A$ jest hermitowski i $A ^ 2 = I$.
\end{definition}

\begin{fact}
    Jeżeli $A$ jest operatorem hermitowskim, to

    $$
        A ^ {2} = I \Longleftrightarrow \sigma(A) \subseteq \left\{+1,-1\right\}.
    $$
\end{fact}

\subsubsection{Macierze Pauliego}

\begin{definition}[Macierze Pauliego]
    Macierzami Pauliego nazywamy następujące macierze:

    $$
        \sigma_{x} =
        \begin{pmatrix}
            0 & 1 \\
            1 & 0
        \end{pmatrix}
    $$

    $$
        \sigma_{y} =
        \begin{pmatrix}
            0 & -i \\
            i & 0
        \end{pmatrix}
    $$

    $$
        \sigma_{z} =
        \begin{pmatrix}
            1 & 0 \\
            0 & -1
        \end{pmatrix}
    $$
\end{definition}

\subsubsection{Operatory von Neumanna}

\begin{definition}[Operator von Neumanna]
    Jeżeli $\hat{v} \in \mathbb{R} ^ 3$ jest wektorem jednostkowym, to operatorem von Neumanna (operatorem pomiaru spinu wzdłuż osi $\hat{v}$) nazywamy operator działający na przestrzeni $V ^ {2} (\mathbb{C})$ dany wzorem

    $$
        \hat{v} \cdot \vec{\sigma} \equiv \sum\limits_{i = 1} ^ {3} v_{i} \sigma_{i} =
        \begin{pmatrix}
            v_3 & v_{1} - i v_{2} \\
            v_{1} + i v_{2} & -v_3
        \end{pmatrix}.
    $$
\end{definition}

Przytoczymy poniżej kilka podstawowych własności operatorów von Neumana.

\begin{theorem}
    Jeżeli $\hat{a} \cdot \vec{\sigma}, \hat{b} \cdot \vec{\sigma}$ są operatorami von Neumanna, to

    $$
        (\hat{a} \cdot \vec{\sigma}) (\hat{b} \cdot \vec{\sigma}) = \left(\hat{a} \cdot \hat{b} \right) I + i \left(\hat{a} \times \hat{b}\right) \cdot \vec{\sigma}.
    $$
\end{theorem}

\begin{fact}
    Operator von Neumanna $\hat{v} \cdot \vec{\sigma}$ jest operatorem hermitowskim. Ponadto łatwo sprawdzić, że $\left(\hat{v} \cdot \vec{\sigma}\right) ^ 2 = I$. Wobec tego, $\hat{v} \cdot \vec{\sigma}$ jest operatorem binarnym.
\end{fact}

\begin{fact}
    Spektrum operatora von Neumanna $\hat{v} \cdot \vec{\sigma}$ to zbiór $\{+1, -1\}$. Ponadto

    $$
    \hat{v} \cdot \vec{\sigma} = \Pi_{+} - \Pi_{-},
    $$
    gdzie $\Pi_{\pm} = \frac{1}{2} (I \pm \hat{v} \cdot \vec{\sigma})$ jest projektorem rzutującym na odpowiednią podprzestrzeń własną.
\end{fact}
