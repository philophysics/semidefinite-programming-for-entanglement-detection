\section{Wykrywanie splątania kwantowego}

\subsection{Kryterium Częściowej Transpozycji}

Okazuje się, że dla układów dwuczęściowych (\ref{equation:bipartite}) ogólnie sprawa rozstrzygnięcia, czy operator gęstości działający na przestrzeni $\mathcal{H}_{AB}$ jest separowalny, jest problemem algorytmicznie NP-trudnym (zob. \cite{rh}). Dla przypadku $n = m = 2$ istnieje jednak bardzo ważne kryterium. Najpierw podamy definicję:

\begin{definition}
    Niech $\mathcal{H}_{A} = \mathcal{H}_{B} = V ^ {d}(\mathbb{C})$ są przestrzeniami Hilberta z ortonormalnymi bazami

    $$
        \{\ket{i ^ {(A)}}\}_{i \in \{0, 1, \ldots, d - 1\}}
    $$
    oraz

    $$
        \{\ket{j ^ {(B)}}\}_{j \in \{0, 1, \ldots, d - 1\}}.
    $$ 

    Niech $A$ jest operatorem liniowym działającym na przestrzeni $\mathcal{H}_{AB}$ i niech w bazie

    $$
        \mathcal{B}_{AB} \equiv \{\ket{i ^ {(A)}} \ket{j ^ {(B)}}\}_{i, j \in \{0, 1, \ldots, d - 1\}}
    $$
    jego elementy macierzowe są równe

    $$
        A \equiv
        \begin{pmatrix}
            A_{00} & A_{01} & \cdots & A_{0, d - 1} \\
            A_{10} & A_{11} & \cdots & A_{1, d - 1 }\\
            \vdots & \vdots & \vdots & \vdots \\
            A_{d - 1, 0} & A_{d - 1, 1} & \cdots & A_{d - 1, d - 1}
        \end{pmatrix},
    $$
    gdzie $A_{ij} \in M_{d \times d}(\mathbb{C})$.

    Częściową transpozycją operatora $A$ ze względu na podukład $B$ nazywamy operator $A^{\Gamma}$, którego elementy macierzowe w bazie $\mathcal{B}_{AB}$ dane są przez:

    $$
        A ^ {\Gamma} \equiv
        \begin{pmatrix}
            A_{00} ^ {T} & A_{01} ^ {T} & \cdots & A_{0, d - 1} ^ {T} \\
            A_{10} ^ {T} & A_{11} ^ {T} & \cdots & A_{1, d - 1} ^ {T} \\
            \vdots & \vdots & \vdots & \vdots \\
            A_{d - 1, 0} ^ {T} & A_{d - 1, 1} ^ {T} & \cdots & A_{d - 1, d - 1} ^ {T}
        \end{pmatrix}.
    $$
\end{definition}

\begin{fact}
    \label{fact:partialtranspose}
    Częściowa transpozycja ze względu na podukład $B$:

    \begin{enumerate}
        \item Zachowuje hermitowskość operatora
        \item Zachowuje ślad operatora
    \end{enumerate}
\end{fact}

\begin{theorem}[Kryterium Częściowej Transpozycji, zob. \cite{rh}]
    \label{theorem:ppt}
    Operator gęstości $\rho$ działający w przestrzeni $2 \otimes 2$ jest stanem separowalnym $\Longleftrightarrow$ operator $\rho ^ {\Gamma}$ jest operatorem gęstości w sensie podanym w charakteryzacji z Twierdzenia \ref{theorem:characterization}.
\end{theorem}

\begin{fact}
    Z Wniosku \ref{corollary:iff} i Faktu \ref{fact:partialtranspose} wynika, że aby sprawdzić, że operator $\rho ^ {\Gamma}$ jest operatorem gęstości, wystarczy pokazać, że jego spektrum jest nieujemne.
\end{fact}

W przypadku operatorów gęstości na układach złożonych wprowadza się też pojęcie \textit{operatora zredukowanego do podukładu}, analogiczne do pojęcia brzegowej gęstości prawdopodobieństwa łącznej zmiennej losowej (klasycznej).

\begin{definition}
    Dla danego łącznego operatora gęstości $\rho ^ {(AB)}$ jego śladem częściowym ze względu na podukład B nazywa się operator (okazuje się, że jest on operatorem gęstości)

    \begin{equation}
        \label{equation:partialtrace}
        \rho ^ {(A)} \equiv \textbf{Tr}_B[\rho ^ {(AB)}],
    \end{equation}
    gdzie działanie $\textbf{Tr}_B$ nazywamy śladowaniem częściowym ze względu na podukład B i dla układów dwuczęściowych oraz operatora gęstości postaci (\ref{equation:complexmixed}) definiuje się je następująco:

    $$
        \textbf{Tr}_{B}[\rho ^ {(AB)}] \equiv \sum\limits_{i, k \in [n_{A}]; j, l \in [n_{B}]} \rho_{\stackrel{ij}{kl}} ^ {(AB)} \ket{a_{i}} \bra{a_{k}} \textbf{Tr}[\ket{b_{j}} \bra{b_{l}}]
    $$

    \begin{equation}
        \label{equation:partial}
        = \sum\limits_{i, k \in [n_{A}]; j, l \in [n_{B}]} \rho_{\stackrel{ij}{kl}} ^ {(AB)} \ket{a_{i}} \bra{a_{k}} \bra{b_{l}}\ket{b_{j}}.
    \end{equation}
\end{definition}

Operator $\rho ^ {(A)}$ powstały z $\rho ^ {(AB)}$ w wyniku śladowania ze względu na podukład $B$ opisuje podukład $A$, tzn. jest jego operatorem gęstości.

\begin{fact}
    Jeżeli stan $\rho ^ {(AB)}$ jest stanem produktowym (\ref{equation:mixedproduct}), to

    \begin{equation}
        \textbf{Tr}_B[\rho ^ {(A)} \otimes \rho ^ {(B)}] = \rho ^ {(A)}.
    \end{equation}
\end{fact}

\begin{proof}
    Niech

    $$
        \rho ^ {(A)} = \sum\limits_{i} p_{i} \ket{\psi_{i} ^ {(A)}} \bra{\psi_{i}^{(A)}}
    $$
    oraz

    $$
        \rho ^ {(B)} = \sum\limits_{j} q_{j} \ket{\psi_{j}^{(B)}} \bra{\psi_{j}^{(B)}}.
    $$

    Wtedy

    $$
        \textbf{Tr}_B\left[\left(\sum\limits_{i} p_{i} \ket{\psi_{i}^{(A)}} \bra{\psi_{i}^{(A)}}\right) \otimes \left(\sum\limits_{j} q_{j} \ket{\psi_{j}^{(B)}} \bra{\psi_{j}^{(B)}}\right)\right]
    $$

    $$
        = \textbf{Tr}_B\left[\sum\limits_{ij} p_{i} q_{j} \ket{\psi_{i}^{(A)}} \bra{\psi_{i}^{(A)}} \otimes \ket{\psi_{j}^{(B)}} \bra{\psi_{j}^{(B)}}\right]
    $$

    $$
        = \sum\limits_{ij} p_{i} q_{j} \ket{\psi_{i}^{(A)}} \bra{\psi_{i}^{(A)}} \underbrace{\bra{\psi_{j}^{(B)}}\ket{\psi_{j}^{(B)}}}_{1}
    $$

    $$
        = \left(\sum\limits_{i} p_{i} \ket{\psi_{i}^{(A)}} \bra{\psi_{i}^{(A)}}\right) \underbrace{\left(\sum\limits_{j} q_{j}\right)}_{1} = \rho ^ {(A)}.
    $$
\end{proof}
