\section{Wprowadzenie do programowania półokreślonego}

\begin{definition}[Programowanie półokreślone -- postać pierwotna]
    Ogólne zagadnienie programowania półokreślonego w postaci pierwotnej definiuje się jako

    $$
        \begin{cases}
            \max \textbf{Tr}[CX] \\
            \text{ze względu na:} \\
            \bullet \textbf{Tr}[A_{i} X] = b_{i}, i = 1, \ldots, p \\
            \bullet X \geq 0
        \end{cases}
    $$
    gdzie

    \begin{itemize}
        \item $X \in M_{n \times n}(\mathbb{R})$ jest macierzą symetryczną traktowaną jako zmienna
        \item $C, A_{i} \in M_{n \times n}(\mathbb{R}), i = 1, \ldots , p$ są danymi macierzami symetrycznymi
        \item $b_{i} \in \mathbb{R}, i = 1, \ldots, p$ są danymi liczbami
    \end{itemize}
\end{definition}

\begin{definition}[Programowanie półokreślone -- postać dualna]
    $$
        \begin{cases}
            \min c ^ {T} x \\
            \text{ze względu na:} \\
                \bullet \text{      } F(x) \leq 0 \\
                \bullet \text{      } A x = b,
        \end{cases}
    $$
    gdzie

    $$
        F(x) = G + \sum \limits_{i = 1}^{m} x_{i} F_{i}
    $$
    dla

    $$
        x \in \mathbb{R} ^ m; c \in \mathbb{R} ^ {m}; G, F_{1}, \ldots, F_{m} \in \textbf{S} ^ {n}
    $$
\end{definition}
